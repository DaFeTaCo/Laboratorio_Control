%Laboratorio de control, UDEA, Bioingeniería, 2025-2

\documentclass[journal]{IEEEtran} 

\IEEEoverridecommandlockouts

% *** Package
% *** Graphics package
\usepackage{graphicx}

% *** Math package 
\usepackage{amsmath}

% *** Float packages
\usepackage{float}

\usepackage{caption}
\captionsetup{
    justification=centering, % centra el caption
    font=footnotesize          % mantiene el tamaño pequeño del documento
}


\def\BibTeX{{\rm B\kern-.05em{\sc i\kern-.025em b}\kern-.08em
    T\kern-.1667em\lower.7ex\hbox{E}\kern-.125emX}}


    
\begin{document}

% paper title

\title{Sistemas Descritos en Función de Transferencia en S y Z}


\author{%%%% author names
    \IEEEauthorblockN{1\textsuperscript{st} Juan Esteban Campillo Zuluaga}% first author
    , \IEEEauthorblockN{2\textsuperscript{nd} Isabela Trujillo Betancourt},
    % delete this line if not needed
     \IEEEauthorblockN{3\textsuperscript{nd} Daniel Felipe Tamayo Cortes            }
    %%%% author affiliations
    \IEEEauthorblockA{\textit{Universidad de Antioquia Facultad de ingeniería, Bioingeniería, Teoría de control}}\\% first 
    
    %%%% corresponding author contact details
    \IEEEauthorblockA{juan.campillo@udea.edu.co, isabela.trujillob@udea.edu.co, danielf.tamayo@udea.edu.co}
}

\maketitle

\section{Introducción}

En el estudio de sistemas dinámicos, las funciones de transferencia constituyen una herramienta fundamental para describir matemáticamente la relación entre una entrada y la salida de un sistema. Estas permiten representar fenómenos fisiológicos, eléctricos o mecánicos de manera simplificada, facilitando el análisis en el dominio de Laplace y, posteriormente, en tiempo discreto mediante transformadas en z. En el contexto de la fisiología respiratoria, este enfoque resulta útil para comprender cómo variables como la presión, el volumen y el flujo se relacionan entre sí a través de modelos lineales aproximados.

En esta práctica se aplicaron estos conceptos mediante el uso de Python y la librería python-control, explorando tanto la simulación en tiempo continuo como la conversión de los modelos al dominio discreto. Para ello se trabajó con una entrada senoidal de presión (Pao) y se analizaron las respuestas de volumen y flujo del sistema respiratorio modelado. De manera general, el propósito de la práctica es fortalecer la comprensión de cómo los sistemas dinámicos pueden representarse y estudiarse a través de herramientas computacionales, integrando teoría y simulación para interpretar el comportamiento de un modelo fisiológico simplificado. 


\section{Información previa}

\subsection{Diagramas de bloques}
Los diagramas de bloques son representaciones gráficas que muestran cómo las señales fluyen a través de un sistema dinámico. Cada bloque representa una función matemática, normalmente una función de transferencia o relación entrada–salida, mientras que las flechas indican el sentido de las señales. Se utilizan porque permiten simplificar el análisis de sistemas complejos, visualizar la interconexión de sus componentes y aplicar de manera más clara las técnicas de control.

Lazo abierto: el sistema responde a una señal de entrada sin retroalimentación. Es decir, la salida no se mide ni se compara con la entrada. Ejemplo: encender un ventilador manualmente sin importar la temperatura de la habitación.

Lazo cerrado: la salida se mide y se retroalimenta hacia la entrada para reducir el error. Aquí existe un controlador que ajusta la señal de entrada según la diferencia entre lo deseado y lo obtenido. Ejemplo: un termostato que regula la temperatura de un cuarto activando o apagando el aire acondicionado

\subsection{Compliancia y resistencias en la mecánica respiratoria}



\begin{itemize}
    \item $R_c$: resistencia de la vía aérea central (por ejemplo, tráquea y bronquios principales).
    \item $R_p$: resistencia periférica de las vías aéreas más pequeñas.
    \item $C_L$: compliancia pulmonar (elasticidad del tejido pulmonar).
    \item $C_W$: compliancia de la pared torácica (elasticidad de la caja torácica).
    \item $C_S$: compliancia del sistema en conjunto, que integra pulmones y pared torácica.
\end{itemize}


La compliancia es la medida de la distensibilidad o capacidad de un sistema respiratorio para expandirse frente a un cambio de presión. Matemáticamente, se define como:

\begin{equation}
    C = \frac{\Delta V}{\Delta P}
\end{equation}

donde $\Delta V$ corresponde al cambio de volumen y $\Delta P$ al cambio de presión.

Su relevancia fisiológica radica en que determina la facilidad con que los pulmones y la caja torácica se expanden durante la respiración. Una compliancia reducida indica rigidez (como en la fibrosis pulmonar), mientras que una compliancia aumentada puede indicar pérdida de elasticidad (como en el enfisema).


\subsection{Simulación en Python}

\noindent
\textbf{Librerías utilizadas y su propósito:}

\begin{itemize}
    \item \textbf{python-control}: se utiliza para el modelado y análisis de sistemas dinámicos y de control. Permite crear funciones de transferencia, realizar simulaciones de la respuesta a entradas estándar (escalón, impulso, senoidal) y analizar estabilidad.
    
    \item \textbf{SciPy}: incluye herramientas avanzadas de integración numérica, resolución de ecuaciones diferenciales y optimización. Es fundamental para simular dinámicas de sistemas fisiológicos o de ingeniería.
    
    \item \textbf{NumPy}: proporciona estructuras de datos eficientes (como arreglos multidimensionales) y operaciones matemáticas de alto rendimiento. Es la base para el cálculo matricial y vectorial necesario en simulaciones.
    
    \item \textbf{Matplotlib}: se emplea para graficar resultados, como respuestas temporales, diagramas de Bode o espectros de señales. Permite visualizar claramente el comportamiento del sistema simulado.

\end{itemize}


\section{Resultados y discusión}

\subsection{\textbf{Mecánica respiratoria}}
\paragraph{\textbf{Función de transferencia en el dominio continuo (S)}}
La función obtenida para la mecánica respiratoria se evidencia en la ecuación \ref{eq:transferencia}

\begin{equation}
\scalebox{0.98}{$\frac{Q}{P_{ao}} = 
\frac{R_p C_s S^2 + \left(1 + \tfrac{C_s}{C_L} + \tfrac{C_s}{C_w}\right)S}
{R_c R_p C_s S^2
+ \left(R_c + R_p + R_c\tfrac{C_s}{C_L} + R_c\tfrac{C_s}{C_w}\right)S
+ \left(\tfrac{1}{C_L} + \tfrac{1}{C_w}\right)}$}
\label{eq:transferencia}
\end{equation}

El volumen se obtuvo al integrar la función de flujo, ya que el flujo es la derivada del volumen con respecto al tiempo. Esto equivale a multiplicar la función de flujo por 1/s en el dominio de Laplace.\bigskip

\paragraph{\textbf{Simulación del modelo respiratorio en condiciones normales y patológicas}}

En primer lugar, se simuló el sistema respiratorio en condiciones normales como se evidencia en la figura \ref{fig:simulacion_normal}, en la cual muestran los resultados de simulación obtenidos para el flujo de aire Q(t), y del volumen V(t) en respuesta a una presión de entrada Pao(t) (excitación sinusoidal) con una amplitud de 2,5 cm de H$_2$O a 15 respiraciones por minuto, lo cual corresponde a la frecuencia normal de respiración en reposo \cite{khoo2000physiological}. Como se puede observar, la onda correspondiente al volumen se encuentra más en fase con Pao, mientras que el flujo de aire muestra un adelanto de fase con Pao. Adicionalmente, se evidencia que el valor máximo alcanzado por la onda de flujo es de ~0,5, mientras que para el volumen es de ~0,4.

\begin{figure}[h!]
    \centering
    \includegraphics[width=1\linewidth]{Comportamiento-Practica1}
    \caption{Dinámica predicha del flujo de aire Q(t) y del volumen V(t) en respuesta a una excitación sinusoidal de Pao (amplitud = 2.5 cm H$_2$O) a 15 respiraciones por minuto}
    \label{fig:simulacion_normal}
\end{figure}

En segundo lugar, se procedió a simular el modelo respiratorio en tres diferentes casos de patologías, como se evidencia en la figura \ref{fig:comparacion}. 

\begin{figure}[H]
    \centering
    \includegraphics[width=1\linewidth]{Comparacion_Prac1.png}
    \caption{Comparación entre un paciente sano vs pacientes con patológias en su volumen y flujo de aire}
    \label{fig:comparacion}
\end{figure}

A continuación se expondrá cada patología simulada:\bigskip

•	\textbf{Obstrucción de la vía respiratoria central}:\\
La obstrucción de la vía respiratoria central (VRC) ocurre cuando hay un estrechamiento o bloqueo en estructuras como la tráquea, laringe o faringe. Esto impide el flujo normal de aire hacia los pulmones y puede ser potencialmente mortal si no se trata rápidamente \cite{mountsinai2023}. 

Por tal motivo, la resistencia de la vía respiratoria central se incrementa de manera significativa debido a la restricción del flujo aéreo, lo que ocasiona una disminución tanto del flujo inspiratorio como del volumen. En el modelo de circuito, este fenómeno se traduce en un aumento de la resistencia central (Rc). Para la simulación, se empleó un valor de 10, lo que corresponde a una enfermedad en grado 2 \cite{nguyen2010respiratory}.

Por ende, observamos en la figura \ref{fig:comparacion} como la curva tanto del volumen como del flujo disminuye su amplitud, lo cual indica que el pulmón no logra llenarse de manera adecuada. \bigskip

•	\textbf{Obstrucción de la vía respiratoria periférica}:\\
La vía respiratoria periférica incluye los bronquios pequeños, bronquiolos y alvéolos. La obstrucción en esta zona suele ser progresiva y difusa, y está asociada a enfermedades como:
\begin{itemize}
    \item EPOC (Enfermedad Pulmonar Obstructiva Crónica): incluye bronquitis crónica y enfisema \cite{nguyen2010respiratory}\cite{contreras2000epoc}
    \item Asma bronquial: obstrucción reversible por hiperreactividad bronquial \cite{nguyen2010respiratory}
\end{itemize}

En estas enfermedades se produce una inflamación crónica, lo cual provoca un engrosamiento de la pared bronquiolar. Esto hace que la resistencia aumente, ya que hay una oposición a la entrada del flujo aéreo, por tal motivo para simular esta patología se modificó la variable relacionada a la resistencia periférica (Rp), aumentandola a un valor de 10 para simular un paciente en grado 2 de la enfermedad \cite{nguyen2010respiratory}\cite{contreras2000epoc}. 

En la figura \ref{fig:comparacion}, se puede observar cómo la curva tanto del volumen como del flujo disminuyen en amplitud, teniendo un comportamiento similar al paciente que sufre de obstrucción en la vía respiratoria central, lo cual nos podría indicar que ambas enfermedades influyen de manera similar en el paciente; por consiguiente, se podría considerar que ambas resistencias inciden de forma semejante en la función de transferencia. \bigskip

•	\textbf{Restricción torácica}:\\
La restricción torácica forma parte del grupo de enfermedades pulmonares restrictivas extrapulmonares, donde el problema no está en el parénquima pulmonar sino en la estructura de la caja torácica o en los músculos respiratorios. Provocando una limitación mecánica para la expansión del tórax, lo que reduce la capacidad de los pulmones para llenarse de aire \cite{resmed2025restrictive}. 

Por tal motivo, hay una reducción de la distensibilidad torácica, lo que provoca un menor volumen inspirado. Esto se representó en la simulación variando el término correspondiente a la compliancia de la pared pectoral (Cw), disminuyendolo a 0.03 \cite{nguyen2010respiratory}.

En la figura \ref{fig:comparacion}, vemos como este valor afecta las curvas tanto del flujo como del volumen disminuyendo su pendiente de manera mucho más significativa que las otras dos enfermedades propuestas anteriormente lo que sugiere que la mecánica respiratoria depende más de los efectos de la compliancia.. Esta disminución en las amplitudes indica que el volumen inspirado es menor y el flujo es reducido debido a la rigidez de la caja torácica, cumpliendo con lo esperado en esta enfermedad. \bigskip


\paragraph{\textbf{Conversión de la función de transferencia al dominio discreto (Z)}}

Para la selección del tiempo de muestreo ($T_s$), se consideró la frecuencia de 15 respiraciones por minuto utilizada en la función de transferencia, lo cual equivale a una frecuencia de 0,25 Hz ($f$) o $\pi/2$ rad/s, como se muestra en la ecuación \ref{eq:freq}.

\begin{equation}
    f = \frac{15}{60} = 0.25 \,\text{Hz} = \frac{\pi}{2} \,\frac{\text{rad}}{\text{s}} \approx 1.57 \,\frac{\text{rad}}{\text{s}}
    \label{eq:freq}
\end{equation}

Este valor es relevante para la aplicación del teorema de Nyquist, el cual establece que una señal en tiempo continuo puede reconstruirse de manera exacta a partir de sus muestras siempre que la frecuencia de muestreo sea, como mínimo, el doble de la frecuencia máxima presente en dicha señal. De este modo, se evita la pérdida de información o distorsiones asociadas al aliasing \cite{academialab2025nyquist}.

De acuerdo con este principio, la frecuencia mínima de muestreo requerida para la correcta reconstrucción de la señal sería de 0,5 Hz. Sin embargo, en la práctica se recomienda emplear valores significativamente superiores, con el fin de capturar variaciones más sutiles y reducir posibles errores de reconstrucción. Por esta razón, en lugar de utilizar el doble de la frecuencia máxima, se decidió trabajar con un factor de 40 veces, obteniendo así una frecuencia de muestreo de 10 Hz, lo que corresponde a un período de muestreo de 0,1 s.

La discretización se realizó mediante dos métodos distintos para su posterior comparación:

\begin{itemize}
    \item \textbf{Zero-Order Hold (ZOH):} Este método asume que la señal de entrada se mantiene constante durante cada intervalo de muestreo \cite{mathworks_continuous_discrete_nodate}.
    \item \textbf{Tustin (transformada bilineal):} Este método utiliza una aproximación trapezoidal para integrar la señal \cite{mathworks_continuous_discrete_nodate}.
\end{itemize}

Las gráficas comparativas de las respuestas de los sistemas se muestran en la figura \ref{fig:simulacion_P4}. 

\begin{figure}[h!]
    \centering
    \includegraphics[width=1\linewidth]{Comparacion_P4.png}
    \caption{Comparacion entre las respuestas del sistema continuo vs sistemas discretos (ZOH y Tustin) en respuesta a una excitación sinusoidal.}
    \label{fig:simulacion_P4}
\end{figure}

La respuesta ZOH se desfasa de la respuesta del sistema continuo porque el método de retención de orden cero mantiene el valor de la señal constante entre las muestras. Esto crea un efecto escalonado, como una serie de bloques, que introduce un retraso o desfase en la respuesta. Este retraso adicional tiene un impacto significativo en la estabilidad y el rendimiento del sistema en lazo cerrado, lo que se manifiesta como picos más altos en la respuesta transitoria.
Mientras que la respuesta Tustin ofrece una mejor representación del sistema continuo ya que en lugar de mantener el valor constante, el método asume que la señal cambia linealmente entre las muestras.


\subsection{\textbf{Análisis del sistema continuo vs. sistemas discretos}}

Se obtuvieron las funciónes de transferencia equivalentes en el dominio discreto (Z) a partir de la función de transferencia continua G(s) (ecuación \ref{eq:F_transferencia}). 

\begin{equation}
    G(s) = \frac{1}{s^2 + 3s + 2}
    \label{eq:F_transferencia}
\end{equation}


\subsubsection{\textbf{Función de transferencia en el dominio discreto}}

Las siguientes son las funciones de transferencia en el dominio Z para los sistemas discretos de flujo y volumen, utilizando los métodos de discretización ZOH y Tustin.

\paragraph{Sistema Discreto del Flujo (ZOH)}

$$H_Z(z) = \frac{1 - 1.166z^{-1} + 0.166z^{-2}}{1 - 0.521z^{-1}}$$

\paragraph{Sistema Discreto del Flujo (Tustin)}

$$H_Z(z) = \frac{0.524 - 0.048z^{-1} - 0.476z^{-2}}{1 + 0.429z^{-1} - 0.476z^{-2}}$$

\paragraph{Sistema Discreto del Volumen (ZOH)}

$$H_Z(z) = \frac{0.051 - 0.051z^{-1} + 0.000z^{-2}}{1 - 1.521z^{-1} + 0.521z^{-2}}$$

\paragraph{Sistema Discreto del Volumen (Tustin)}

$$H_Z(z) = \frac{0.026 + 0.024z^{-1} - 0.026z^{-2} - 0.024z^{-3}}{1 - 0.571z^{-1} - 0.905z^{-2} + 0.476z^{-3}}$$

Se procedió a simular las respuestas para cada sistema, tanto en lazo abierto como en lazo cerrado. Se utilizaron una entrada de escalón unitario y una entrada de impulso unitario.


\subsubsection{\textbf{Respuesta del sistema discreto ante diferentes entradas}}

A medida que el tiempo de muestreo ($T_s$) aumenta, el sistema discreto pierde información entre las muestras, lo que provoca una degradación en la respuesta del sistema. Esto se manifiesta en oscilaciones, mayor sobreimpulso y una divergencia notable con respecto a la respuesta del sistema continuo original.  Este retraso de muestreo es una consecuencia directa de la digitalización, ya que al mantener el valor constante entre muestras, el sistema discreto va siempre un paso atrás.

En lazo abierto (LA) y lazo cerrado (LC), un ($T_s$) grande compromete la capacidad del sistema discreto para simular con precisión el comportamiento del sistema continuo.

\subsubsection{\textbf{Respuesta del sistema original continuo ante diferentes entradas}}

Las respuestas del sistema continuo se utilizan como referencia para comparar el desempeño del sistema discreto. El texto indica que, en general, las respuestas del sistema discreto divergen cada vez más de la respuesta del sistema continuo a medida que aumenta el tiempo de muestreo.  Además, el párrafo no menciona cómo se comporta el sistema continuo en lazo abierto versus lazo cerrado, ya que la comparación se centra en cómo el sistema discreto se desvía del continuo.


\begin{figure}[h!]
    \centering
    \includegraphics[width=1\linewidth]{Comparacion_P5.png}
    \caption{Comparacion entre las respuestas de un segundo sistema en su forma continua vs sistema discreto variando el periodo de muestreo.}
    \label{fig:simulacion_P5}
\end{figure}





\section{Conclusiones}

La precisión de la discretización depende fundamentalmente del tiempo de muestreo ($T_s$). Para obtener una representación fiel del sistema continuo en el dominio discreto, es esencial utilizar un tiempo de muestreo lo suficientemente pequeño.

Para el modelo empleado, la obstrucción central y periférica tienen efectos equivalentes, lo que implicaría que la respuesta de la vía respiratoria central es similar a la de la vía respiratoria periférica. Sin embargo, esto hace parte de las simplificaciones que se asumen dentro del modelo.

Los cambios en los distintos parámetros de modelado pueden ayudar a comprender mejor el comportamiento e implicaciones de los cambios en la resistencia y compliancia para diferentes patologías.


\bibliographystyle{ieeetr}   % estilo de bibliografía 
\bibliography{bibliografia-Pract1}  % nombre del archivo .bib SIN la extensión

\end{document}

